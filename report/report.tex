

\documentclass[conference,compsoc]{IEEEtran}


% *** CITATION PACKAGES ***
%
\ifCLASSOPTIONcompsoc
  % IEEE Computer Society needs nocompress option
  % requires cite.sty v4.0 or later (November 2003)
  \usepackage[nocompress]{cite}
\else
  % normal IEEE
  \usepackage{cite}
\fi


% *** GRAPHICS RELATED PACKAGES ***
%
\ifCLASSINFOpdf
  % \usepackage[pdftex]{graphicx}
  % declare the path(s) where your graphic files are
  % \graphicspath{{../pdf/}{../jpeg/}}
  % and their extensions so you won't have to specify these with
  % every instance of \includegraphics
  % \DeclareGraphicsExtensions{.pdf,.jpeg,.png}
\else
  % or other class option (dvipsone, dvipdf, if not using dvips). graphicx
  % will default to the driver specified in the system graphics.cfg if no
  % driver is specified.
  % \usepackage[dvips]{graphicx}
  % declare the path(s) where your graphic files are
  % \graphicspath{{../eps/}}
  % and their extensions so you won't have to specify these with
  % every instance of \includegraphics
  % \DeclareGraphicsExtensions{.eps}
\fi




% correct bad hyphenation here
\hyphenation{op-tical net-works semi-conduc-tor}

% for superscripts like first, second...
\usepackage[super]{nth}

\begin{document}
%
% paper title
% Titles are generally capitalized except for words such as a, an, and, as,
% at, but, by, for, in, nor, of, on, or, the, to and up, which are usually
% not capitalized unless they are the first or last word of the title.
% Linebreaks \\ can be used within to get better formatting as desired.
% Do not put math or special symbols in the title.
\title{Big Data Processing Systems\\\nth{1} Project}


% author names and affiliations
% use a multiple column layout for up to three different
% affiliations
\author{\IEEEauthorblockN{Helder Rodrigues}
\IEEEauthorblockA{MAEBD\\FCT UNL PT\\Email: harr@campus.fct.unl.pt}

}




% make the title area
\maketitle

% As a general rule, do not put math, special symbols or citations
% in the abstract
\begin{abstract}
This document presents a set of experiments with the aim of showing the capabilities of 2 Big Data technologies, MapReduce and Spark RDD, for extracting insights by processing a great volume of data.
\end{abstract}

% no keywords




% For peer review papers, you can put extra information on the cover
% page as needed:
% \ifCLASSOPTIONpeerreview
% \begin{center} \bfseries EDICS Category: 3-BBND \end{center}
% \fi
%
% For peerreview papers, this IEEEtran command inserts a page break and
% creates the second title. It will be ignored for other modes.
\IEEEpeerreviewmaketitle




\section{Introduction}
% no \IEEEPARstart
We will go trough the Project 1 statement\cite{IEEEhowto:bdpsp1} analysing the  information about taxi rides in some city for better understanding the behavior of the community and help taxi drivers maximize their profit. We will use Big Data technology during our experiments, MapReduce and Spark RDD, and take conclusions about the analysis methods. We used python as programming language for all experiments.

\hfill hr

\hfill November 2, 2020

% \subsection{Subsection Heading Here}
% Subsection text here.


% \subsubsection{Subsubsection Heading Here}
% Subsubsection text here.

\section{Dataset}
The Dataset consists in a collection of taxi trip records in some city, including fields for capturing pick-up and drop-off dates/times, pick-up and drop-off locations and trip distances.

\section{MapReduce}
We used MapReduce to create indexes for answering the following queries:
\subsection{How many trips were started in each year present in the data set?}
We developed a pair of Map and Reduce programs for running on top of Hadoop. \par The Mapper processes each line of the provided CSV file. It cleans its contents and extracts the respective year. As the objective is to count the trips per year and because each line represents 1 trip, it outputs the year as key \textit{(K)} and 1 \textit{(one trip)} as value \textit{(V)}. \par The Reducer will receive a sequence of the records extracted from the mappers, ordered by key. The objective is to sum all values for the same key and output a record with the following format: \textit{(year,sum of trips)}. Table \ref{output_1_1} shows an example of the obtained data.
\begin{table}[!t]
\renewcommand{\arraystretch}{1.3}
\caption{Results of 1.1}
\label{output_1_1}
\centering
\begin{tabular}{c||c}
\hline
\bfseries Year & \bfseries Number of trips\\
\hline\hline
2013 & 54409\\
2014 & 74753\\
... & ...\\
2019 & 32797\\
2020 & 6829 \\
\hline
\end{tabular}
\end{table}

\subsection{For each of the 24 hours of the day, how many taxi trips there were, what was their average trip miles and trip total cost?}
We developed a pair of Map and Reduce programs for running on top of Hadoop. \par Here as the objective is to calculate an average, the Mapper will extract the trip miles and trip values from each line. The Reducer will sum the number of lines together with the sum of trip miles and trip cost. After iterating each key (hour) it outputs the average values by dividing the sums of trip miles and cost by the trip counts. Table \ref{output_1_2} shows an example of the obtained data.
\begin{table}[!t]
\renewcommand{\arraystretch}{1.3}
\caption{Results of 1.2}
\label{output_1_2}
\centering
\begin{tabular}{c||c|c|c}
\hline
\bfseries Hour & \bfseries Num trips & \bfseries Avg Miles & \bfseries Avg Cost\\
\hline\hline
01 AM&11164&2.46&13.11\\
01 PM&20177&3.38&15.54\\
02 AM&8832&2.35&12.71\\
...&...&...&...\\
12 AM&13542&2.83&14.08\\
12 PM&19872&3.38&15.44\\
\hline
\end{tabular}
\end{table}

\subsection{For each of the 24 hours of the day, which are the 5 most popular routes according to the the total number of taxi trips? Also report and the average fare.}
This problem wasn't trivially solvable with a single pair of MapReduce. Thus we created 2 MapReduce pairs: The first one behaves like the previous exercise, just adding the route to the grouping key. The second Mapper calculates a reverse order ranking, by subtracting the trip count from a very large number and concatenates it at the end of the key. The infrastructure sorts then the records by key (hour / route / order) and passes them to the second Reducer that simply prints out the first 5 records for each key. Table \ref{output_1_3} shows an example of the obtained data. \par Conversely we could opt by output the original counts from the Mapper and instantiate a specific Comparator by dependency injection in the infrastructure, for reversing the comparison function. However after testing this approach, by using \textit{KeyFieldBasedComparator} class we realised that the performance was substantially reduced as by Table \ref{perf_1_4}.

\begin{table}[!t]
\renewcommand{\arraystretch}{1.3}
\caption{Results of 1.3}
\label{output_1_3}
\centering
\begin{tabular}{c|c||c|c|c}
\hline
\bfseries Hour & \bfseries Route & \bfseries NTrps & \bfseries AMiles & \bfseries ACost\\
\hline\hline

01 AM&17031081700 17031081700&96&0.46&6.91	\\
01 AM&17031081700 17031081800&73&0.54&6.96	\\
...&...&...&...&...	\\
08 PM&17031980000 17031980000&115&1.86&12.17\\
11 PM&17031839100 17031320100&64&0.69&7.08	\\

\hline
\end{tabular}
\end{table}

\begin{table}[!t]
\renewcommand{\arraystretch}{1.3}
\caption{Performance comparison 1.3}
\label{perf_1_4}
\centering
\begin{tabular}{c||c|c}
\hline
\bfseries Time & \bfseries Default Comparator & \bfseries KeyFieldBasedComparator \\
\hline\hline

 real & 0m2.376s           & 0m4.661s                \\
 user & 0m2.600s           & 0m6.075s                \\
 sys  & 0m1.298s           & 0m2.320s \\
\hline
\end{tabular}
\end{table}

\section{PySpark}
We used PySpark RDD to create indexes for answering the following queries:
\subsection{What is the accumulated number of taxi trips per month?}
We developed a small Python programs that instantiates the Spark context for obtaining this insight.\par It starts by reading the dataset,  followed by a filter for eliminating empty lines. This out-of-the-box filter function is a big simplification when comparing with MapReduce, where we would need to create Mapper and a Reducer to accomplish the same result.  \par
The next 2 steps \textit{map} and \textit{reduceByKey} behave mostly like a MapReduce pair. For the \textit{reduceByKey} we use an operation \textit{add} that receives 2 values and calculates the addition. Table \ref{output_2_1} shows an example of the obtained data.

\begin{table}[!t]
\renewcommand{\arraystretch}{1.3}
\caption{Results of 2.1}
\label{output_2_1}
\centering
\begin{tabular}{c||c}
\hline
\bfseries Month & \bfseries Number of trips\\
\hline\hline

7 &32141\\
3 &35260\\
... & ...\\
4 &32884\\
9 &31466\\
\hline
\end{tabular}
\end{table}

\subsection{For each pickup region, report the list of unique dropoff regions?}
We had here the opportunity to use another out-of-the-box function: \textit{distinct}. In a single line we can avoid to code a MapReduce pair of programs. \par Another interesting construction was the \textit{groupByKey} function that allows us to collect all the RDD values in a single list without the need for coding any specific iterator.  Table \ref{output_2_2} shows an example of the obtained data.

\begin{table}[!t]
\renewcommand{\arraystretch}{1.3}
\caption{Results of 2.2}
\label{output_2_2}
\centering
\begin{tabular}{c||c}
\hline
\bfseries Pickup region & \bfseries Dropoff regions\\
\hline\hline

17031770202 &['17031770202']\\
17031020301 &['17031020802', '17031020301']\\
17031837100 &['17031243000', '17031320400', '17031837100']\\
17031030104 &['17031030104', '17031980000', '17031280100']\\
17031040300 &['17031320400', '17031040300']\\
... &...\\
\hline
\end{tabular}
\end{table}

\subsection{What is the expected duration and distance of a taxi ride, given the pickup region ID, the weekday (0=Monday, 6=Sunday) and time in format “hour AM/PM”?}

\begin{table}[!t]
\renewcommand{\arraystretch}{1.3}
\caption{Results of 2.3}
\label{output_2_3}
\centering
\begin{tabular}{c||c|c}
\hline
\bfseries Pickup\_Weekday\_Time & \bfseries Avg Duration & \bfseries Avg Distance\\
\hline\hline
17031081000\_6\_09PM & 537.40 & 1.10\\
17031838200\_4\_09AM & 425.33 & 1.63\\
... & ... & ...\\
17031062600\_5\_08PM & 1057.25 & 2.33\\
17031062100\_3\_09AM & 956.00 & 6.90\\
\hline
\end{tabular}
\end{table}
\section{Conclusion}
The conclusion goes here.
\subsection{MapReduce}
\begin{itemize}
\item The Mappers and reducers are different programs.
\item Control of program parameters like sorting from outside the program
\end{itemize}
\subsection{PySpark}
\begin{itemize}
\item PySpark RDD has a good set of out-of-the-box functions for helping us massaging the data in a performant and scalable way without the need of writing custo MapReduce implementations. The code becomes more readable and the implementation is scalable and performant.
\item Lazy
\end{itemize}





% conference papers do not normally have an appendix



% use section* for acknowledgment
\ifCLASSOPTIONcompsoc
  % The Computer Society usually uses the plural form
  \section*{Acknowledgments}
\else
  % regular IEEE prefers the singular form
  \section*{Acknowledgment}
\fi


The authors would like to thank...





% trigger a \newpage just before the given reference
% number - used to balance the columns on the last page
% adjust value as needed - may need to be readjusted if
% the document is modified later
%\IEEEtriggeratref{8}
% The "triggered" command can be changed if desired:
%\IEEEtriggercmd{\enlargethispage{-5in}}

% references section

% can use a bibliography generated by BibTeX as a .bbl file
% BibTeX documentation can be easily obtained at:
% http://mirror.ctan.org/biblio/bibtex/contrib/doc/
% The IEEEtran BibTeX style support page is at:
% http://www.michaelshell.org/tex/ieeetran/bibtex/
%\bibliographystyle{IEEEtran}
% argument is your BibTeX string definitions and bibliography database(s)
%\bibliography{IEEEabrv,../bib/paper}
%
% <OR> manually copy in the resultant .bbl file
% set second argument of \begin to the number of references
% (used to reserve space for the reference number labels box)
\begin{thebibliography}{1}

\bibitem{IEEEhowto:bdpsp1}
J. Lourenço, \emph{Big Data Processing Systems — Project nº 1}, 2020/21.\hskip 1em plus
  0.5em minus 0.4em\relax DI, FCT, UNL, PT.

\bibitem{IEEEhowto:test}
Thomas Erl, Wajid Khattak and Paul Buhler \emph{Big Data Fundamentals}, 2016\hskip 1em plus 0.5em minus 0.4em\relax Prentice Hall.

\end{thebibliography}




% that's all folks
\end{document}


